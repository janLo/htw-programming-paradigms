\documentclass[final,a4paper,11pt,notitlepage,halfparskip]{scrreprt}

\usepackage[german,ngerman]{babel}
\usepackage[utf8]{inputenc}
\usepackage[T1]{fontenc}
\usepackage[babel,german=quotes]{csquotes}
%\usepackage{fancybox}
%\usepackage{color}
\usepackage{xcolor}
\usepackage{hyperref}
%\usepackage{floatflt}
\usepackage{graphicx}
\usepackage{amsmath}
\usepackage{amssymb}
\usepackage{amsfonts}
\usepackage{listings}

\lstset{language=bash, 
    keywordstyle=\color{blue!80!black!100}, 
    identifierstyle=, 
    commentstyle=\color{green!50!black!100}, 
    stringstyle=\ttfamily, 
    breaklines=true, 
    columns=fixed,
    numbers=left, 
    xleftmargin=20pt,
    framexleftmargin=21pt,
    rulecolor=\color{white!40!black!70},
    backgroundcolor=\color{white!90!black!25},
    numbersep=9pt,
    numberstyle=\tiny, 
    frame=single, 
    % basicstyle=\small\ttfamily\bfseries, % print whole listing small 
    basicstyle=\scriptsize\ttfamily\bfseries, % print whole listing small 
    linewidth=\linewidth,       % Zeilenbreite 
    breaklines=true,
    showstringspaces=false,
    breakatwhitespace=true, %Umbruch an Leerzeichen 
    captionpos=t 
}

\setkomafont{caption}{\footnotesize\linespread{1}\selectfont}
\setlength{\abovecaptionskip}{-0.1cm}
\addto\captionsngerman{\renewcommand\figurename{Abb.}}

\subject{Beleg Programmierparadigmen}
\title{Simulation eine Turingmaschine}
\author{Jan Losinski\\
\small{Mat. Nr. 25517}\\
\small{g08s29}}

\begin{document}

\maketitle

\tableofcontents

\chapter{Implementierung}
\section{Allgemeines}
Eine Turingmaschine ist ein einfaches Konstrukt aus einem (in beide 
Richtungen unendlichem) Band, einem Lese/Schreibkopf, einer Menge von
Zuständen, einem Bandalphabet, einem Eingabealphabet welches eine
Teilmenge des Bandalphabetes ist und einer Menge von Zustandsübergängen.

Dabei ist ein Zustand als Startzustand und (mindestens) einer als
Endzustand definiert, an denen die Turingmaschine ihre Arbeit beginnt
und an dem sie anhält. Das Ergebnis des Laufes steht nach dem Halten, 
die Eingabe vor dem Laufen auf dem Band.

Die Formale Definition einer Turingmaschine $M=(Q, \Sigma, \Gamma,
\delta, q_0, \square, F)$ lautet wie Folgt:

\begin{itemize}
    \item  $Q$ ist die endliche Zustandsmenge.
    \item  $\Sigma$ ist die endliche Zustandsmenge.
    \item  $Gamma \supset \Sigma$ ist das endliche Bandalphabet.
    \item  $\square \in \Gamma\setminus\Sigma$ steht für das leere Feld.
    \item  $F \subseteq Q$ ist die Menge der End- bzw. akzeptierenden Zustände.
    \item  $\delta: (Q \setminus F)\times \Gamma \rightarrow Q \times
	\Gamma \times \{ L, 0, R \}$ ist die Überführungsfunktion.
    \item  $q_0 \in Q$ ist der Anfangszustand.  	
    \item  $q_{e1} \dots q_{en} \in Q$ sind die Endzustände.  	
\end{itemize}

Die Aufgabe besteht darin, die Bestandteile der Turingmaschine in Prolog
zu modellieren. Die Ausführung des Modells soll anschließend simuliert
werden.

Die Modellierung geschieht durch Definition geeigneter Prädikate. So
werden die Zustände durch das Prädikat \texttt{z} definiert, die
Endzustände durch \texttt{ze} usw. Die Definition der konkreten Maschine
steht zudem in einer gesonderten Datei: 
\begin{itemize}
    \item Die Datei \texttt{turing.pl} enthält die Klauseln zur
	Simulation der Maschine
    \item Die Datei \texttt{program.pl} enthält die Klauseln zur
	Definition der zu simulierenden Maschine.	
\end{itemize}
Die beiden Dateien müssen daher beide vor der Simulation in das
Prologsystem geladen werden. Der Vorteil besteht darin, das es
Problemlos möglich ist, die Definition durch eine neue auszutauschen.

\section{Das Band}
Das Band der Turingmaschine ist laut Definition in beide Richtungen
unendlich lang. Der Zugriff auf die einzelnen Speicherzellen des Bandes
ist nur sequentiell möglich. Dies bedeutet, das ein Wahlfreier Zugriff
mittels einer absoluten oder einer Offsetpositionierung nicht
implementiert werde muss.

Auf Grund dieser Eigenschaften wurde eine Implementierung als zwei
unbegrenzte Stapel gewählt. Diese Repräsentieren die beiden Richtungen
des Bandes. Dazu wurden die Klauseln \texttt{myPush} zum hinzufügen
eines Elementes zum Stapel und \texttt{myPop} zum entfernen des obersten
Elementes eines Stapels eingeführt. Sollte einer der beiden Stapel
einmal leer sein, so wird bei \texttt{myPop} das Blank-Symbol als 
aktuelles Element und eine leere Liste als Stapel zurückgegeben.

Auf diesen zwei Klauseln aufbauend wird das Verhalten eines echten
Bandes mit den Klauseln \texttt{myPrev} und \texttt{myNext} simuliert.
Diese setzen den "`Lesekopf"' der simulierten Turingmaschine auf die
vorherige, bzw. die nächste Bandzelle. Realisiert wird dies durch ein
Umverteilen der Elemente. So wird bei \texttt{myPrev} das aktuelle
Element auf den Vorgängerstapel gelegt und das oberste Element des
Nachfolgerstapels als neues aktuelles Element geholt. Bei
\texttt{myNext} geschieht dies entsprechend umgekehrt. Der Vorteil ist
hier der einfache Aufbau und die nicht vorhandene Längenbegrenzung (vom
begrenzen Adressbereich der unterliegenden Hardware einmal abgesehen).

Zusätzlich existiert noch die Klausel \texttt{myMove} mit verschiedenen
Argumenten. Diese dient der direkten Abbildung von Bandaktionen in den
Zustandsübergangsdefinitionen auf die Bandoperationen.

\section{Ausgabe}
Die Ausgabe der Simulation wird durch verschiedene Klauseln realisiert.
Die Klauseln \texttt{myPrintIter}, \texttt{myPrint} und
\texttt{myPrint2} sind für die Ausgabe des aktuellen Bandes zuständig.
In \texttt{myPrintIter} ist die eigentliche Ausgabe der Inhalte des
Bandes definiert. Dabei wird die Stapelliste durchgegangen und Element
für Element durch Komma getrennt ausgegeben. Die Klauseln
\texttt{myPrint} und \texttt{myPrint2} dienen der Vorbereitung der
Stapel für \texttt{myPrintIter}. So muss zum Beispiel der Inhalt des
Vorgängerstapel mittels des eingebauten Prädikates \texttt{reverse}
umgekehrt werden. Zudem sorgt \texttt{myPrint2} mittels der Hilfsklausel
\texttt{replaceNonInput} dafür, das nicht zum Eingabealphabet gehörende
Zeichen durch ein Leerzeichen ersetzt werden. Dies ist notwendig, da auf
dem Ausgabeband solche Zeichen nicht auftauchen dürfen.

Die Klausel \texttt{myPrintProgress} dient der Ausgabe des Fortschritts
der Simulation. Sie wird nach jedem Zustandsübergang aufgerufen. Sie
gibt alle Informationen, wie das momentane Band, die Zustände, die
gelesenen und geschriebenen Symbole, etc aus. Am Ende der
Klauseldefinition wird auf die Eingabe eines Bytes gewartet. Dies dient
der Unterbrechung der Simulation bis zum Drücken der "`Enter"' Taste.

\section{Simulation}
Die Simulation besteht aus dem Auswählen des Startzustandes und dem
vollziehen der Zustandsübergänge, bis ein Endzustand erreicht ist. Dies
geschieht durch die Klauseln \texttt{run} und \texttt{start}. 

Die Klausel \texttt{run} vollzieht dabei die eigentlichen Zustandsübergänge
und "`sucht"' dabei den Endzustand. Bei jedem Übergang wird der Aktuelle
Status aller Zeile der Maschine ausgegeben. Wurde der Endzustand
gefunden, so werden verschiedene Informationen, wie die Anzahl der
Schritte und das Ausgabeband ausgegeben.

Die Klausel \texttt{start} prüft das Vorhandensein eines Startzustandes,
eines Anfangsbandes sowie der Position auf dem Anfangsband. Ist alles
vorhanden, so wird das Band "`vorbereitet"'. Dies wird durch die Klauseln
\texttt{prepareBand} und \texttt{prepareBand2} übernommen. Sie füllen
das Band mit dem Inhalt der als Anfangsband angegebenen Liste und setzen
den "`Lesekopf"' aus die angegebene Position. Anschließend wird das
Vollziehen der Zustandsübergänge mit dem Startzustand in Gang gesetzt.

\chapter{Benutzung}

\section{Definition}
Die Turingmaschine muss zuerst definiert werden. Die Definitionen stehen
in der Datei \texttt{program.pl}. Dazu gehören folgende Bestandteile:
\begin{itemize}
    \item Die Bewegungsrichtung des Bandes "`\texttt{r(Zeichen)}"'. Für
	"`Zeichen"' gibt es genau drei Werte: "`\texttt{r}"' für rechts,
	"`\texttt{l}"' für Links und "`\texttt{n}"' um die Position nicht 
	zu ändern.
    \item Das Eingabealphabet "`\texttt{e(Zeichen)}"'. Um zum Beispiel
	"`\texttt{1}"' als Zeichen des Eingabealphabes zu definieren
	muss die Klausel "`\texttt{e(1)}"' definiert werden.
    \item Das Blank-Zeichen \texttt{bl(Zeichen)}, welches ein leeres
	Feld auf dem Band repräsentiert. Soll "`\texttt{\%}"' als
	Blank-Zeichen zu definieren ist die Klausel "`\texttt{bl(\%)}"'
	nötig.	
    \item Das Bandalphabet "`\texttt{b(Zeichen)}"'. Dies besteht aus 
	dem Eingabealphabet und dem Blank-Zeichen. Daher besteht diese 
	Definition aus den zwei Klauseln "`\texttt{b(X) :- e(X).}"' 
	und "`\texttt{b(X) :- bl(X).}"'.
    \item Den Zuständen "`\texttt{z(Zeichen)}"'. Jeder Zustand
	"`\texttt{n}"' muss mit einer Definition "`\texttt{z(n)}"' 
	aufgeführt werden.	
    \item Dem Anfangszustand "`\texttt{za(zustand)}"'. Soll zum 
	Beispiel der Zustand "`\texttt{0}"' (def. als "`\texttt{z(0)}"') 
	als Startzustand definiert werden muss die Klausel
	"`\texttt{za(z(0))}"' in der Definition existieren.
    \item Dem Edzustand "`\texttt{ze(zustand)}"'. Soll zum 
	Beispiel der Zustand "`\texttt{0}"' (def. als "`\texttt{z(0)}"') 
	als Endzustand definiert werden muss die Klausel
	"`\texttt{ze(z(0))}"' in der Definition existieren.
    \item Die Zustandsübergänge 
	"`\texttt{zu(zustd, gelesen, schreiben, bewegen, nextzust)}"'.
	Um zum Beispiel einen Übergang von Zustand "`\texttt{0}"' zu
	Zustand "`\texttt{1}"' bei gelesenem "`\texttt{\%}"' zu
	definieren, bei dem "`\texttt{1}"' geschrieben wird und das Band
	nach Rechts bewegt wird dient die Klausel 
	"`\texttt{zu(z(0),'\%', '1',  r(l), z(1))}"'.
    \item Das Eingabe- oder Anfangsband "`\texttt{ab(Liste)}"'. Eine
	Beispielbelegung sieht we folgt aus:
	"`\texttt{ab(['\%','1','1','\%','\%','1']).}"'	
    \item Die Anfangsposition auf dem Eingabealphabet
	"`\texttt{ap(zahl)}"'. Soll der "`Lesekopf"' Zum start auf dem
	dritten Element stehen ist die Definition "`\texttt{ap(3)}"'
	notwendig.	
\end{itemize}
All diese Definitionen sind notwendig für die Ausführung der Simulation.

\section{Ausführung}
Zur Ausführung der Simulation müssen beide Dateien (\texttt{turing.pl}
mit den Klauseln zur Simulation und \texttt{program.pl} mit den
Definitionen der zu simulierenden Maschine) in das Prologsystem geladen
werden. Als Prologsystem wurde bei der Implementierung SWI-Prolog
gewählt.  

Das Laden der Dateien geschieht im SWI-Prolog durch angabe des
Dateinamens in Eckigen Klammern mit abschließendem Punkt. Anschließend
kann die Simulation mit der Klausel \texttt{start} gestartet werden:
\begin{lstlisting}[language=Prolog]
?- [turing].
% turing compiled 0.01 sec, 0 bytes
true.
?- [program].
% program compiled 0.00 sec, 0 bytes
true.
?- start.
\end{lstlisting}
Danach startet die Simulation, was an der Ausgabe der einzelnen Schritte
erkannt werden kann. Die Ausgabe eines Schrittes sieht wie folgt aus:
\begin{lstlisting}[]
?- start.
Schritt 1:
    Momentaner Zustand: 0; Momentanes Band:
      ..., % , %, %, %, %, [%], %, %, %, %, ...
    Gelesen: %; Schreibe: 1
    Bewege: Links ; Folgezustand: 1
    Resultierendes Band: 
      ..., % , %, %, %, %, 1, [%], %, %, %, ...
fortsetzen mit [Enter]
\end{lstlisting}
Jetzt kann die Simulation Schritt für Schritt durchgeführt werden. Wenn
der Endzustand gefunden wurde, so wird eine abschließende Information
ausgegeben. Dies sieht z.B. wie folgt aus:
\begin{lstlisting}[]
Fertig nach 7 Schritten:
    Endzustand: 4
    Elemente vor der aktuellen Position:  3
    Elemente nach der aktuellen Position: 2
    Resultierendes Band:
       ..., % , %, %, %, %, [1], 1, 1, %, %, ...
    Ausgabeband (Band ohne Nichteingabesymbole):
       [ ,  ,  , [1], 1, 1]
true.
?- 
\end{lstlisting}
Anschließend kann die Simulation erneut gestartet werden. Wenn die
Definition geändert werden soll muss vorher die Datei
\texttt{program.pl} neu geladen werden.

\pagebreak
\begin{appendix}
    \chapter{Quellen}
    \lstinputlisting[caption=\texttt{turing.pl},language=Prolog]{../turing.pl}
    \pagebreak
    \lstinputlisting[caption=\texttt{program.pl},language=Prolog]{../program.pl}
\end{appendix}

\end{document}
